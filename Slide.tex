\documentclass{beamer}
\usepackage[utf8]{inputenc}
\usepackage[spanish]{babel}
\usepackage{multicol} % indice en 2 columnas
\usetheme{Warsaw}
\usecolortheme{crane}
\useoutertheme{shadow}
\useinnertheme{rectangles}

\setbeamertemplate{navigation symbols}{} % quitar simbolitos

\title[Latex]{Aprendiendo Latex}
\subtitle{LaTeX y Git aplicado a la investigación científica\\ (Darwin Aventur)}
\author[Renato L. Ramírez Rivero]
{renatolrr\\
www.renatoramirez.com}

\date{11 al 15 de enero de 2016}

\AtBeginSection{
\begin{frame}
  \frametitle{Indice}
  \tableofcontents[currentsection]  
\end{frame}
}

\AtBeginSubsection{
\begin{frame}
  \frametitle{Indice}
  \tableofcontents[currentsection,currentsubsection]
\end{frame}
}

\begin{document}

\frame{\titlepage}

\begin{frame}
\frametitle{Indice}
\tableofcontents
\end{frame}

\section{Nociones básicas de LaTeX y su funcionamiento}
\subsection{Procesadores de textos y su historia.}

\begin{frame}
  \frametitle{Procesadores de textos y su historia.}
 % \begin{itemize}
  %\item<1->{*}
  %\end{itemize}
\end{frame}

\subsection{Consiguiendo e instalando LaTeX.}
\begin{frame}
  \frametitle{Consiguiendo e instalando LaTeX.}
  \begin{itemize}
  \item<1->{Instalando LaTeX en Windows.}
  \item<2->{¿Problemas con la instalación de LaTeX?.}
   \end{itemize}
\end{frame}


\subsection{Trabajando con LaTeX.}
\begin{frame}
  \frametitle{Trabajando con LaTeX.}
 % \begin{itemize}
  %\item<1->{*}
  %\end{itemize}
\end{frame}

\section{Primeros pasos. Estructura y creación de documentos}
\subsection{El estado mental correcto.}
\begin{frame}
  \frametitle{El estado mental correcto.}
 \begin{itemize}
  \item<1->{El estado mental correcto.}
  \item<2->{Editor de LaTeX Texmaker.}
  %item<3->{*}
  \end{itemize}
\end{frame}

\subsection{Estructura básica de un documento.}
\begin{frame}
  \frametitle{Estructura básica de un documento.}
 % \begin{itemize}
  %\item<1->{*}
  %\end{itemize}
\end{frame}

\subsection{Creando contenido.}
\begin{frame}
  \frametitle{Creando contenido.}
  \begin{itemize}
  \item<1->{Creando contenido.}
  \item<2->{Capítulos y secciones.}
  \item<3->{Entornos.}
  \item<4->{Fórmulas matemáticas.}
  \end{itemize}
\end{frame}

\subsection{Crear un documento básico.}
\begin{frame}
  \frametitle{Procesadores de textos y su historia.}
  \begin{itemize}
  \item<1->{Crear un documento básico.}
  \item<2->{*}
  %item<3->{*}
  \end{itemize}
\end{frame}

\section{Avanzando en contenido y forma}
\subsection{Acentos y UTF-8.}
\begin{frame}
  \frametitle{Acentos y UTF-8.}
  \begin{itemize}
   \item<1->{Acentos y UTF-8.}
   \item<2->{Caracteres especiales.}
  \end{itemize}
\end{frame}

\subsection{Tipos de letra.}
\begin{frame}
  \frametitle{Tipos de letra.}
  \begin{itemize}
  \item<1->{Tipos de letra.}
  \item<2->{Más tipos de letra.}
  \item<3->{Cambiado el tamaño de la letra.}
  \item<4->{Unidades de medida.}
  \end{itemize}
\end{frame}

\section{Ampliación de entornos. Fórmulas matemáticas y símbolos especiales.}

\section{Bibliografía.}

\end{document}